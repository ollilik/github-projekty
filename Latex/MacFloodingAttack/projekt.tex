%==============================================================================
% tento soubor pouzijte jako zaklad
% this file should be used as a base for the thesis
% Autoři / Authors: 2008 Michal Bidlo, 2019 Jaroslav Dytrych
% Kontakt pro dotazy a připomínky: sablona@fit.vutbr.cz
% Contact for questions and comments: sablona@fit.vutbr.cz
%==============================================================================
% kodovani: UTF-8 (zmena prikazem iconv, recode nebo cstocs)
% encoding: UTF-8 (you can change it by command iconv, recode or cstocs)
%------------------------------------------------------------------------------
% zpracování / processing: make, make pdf, make clean
%==============================================================================
% Soubory, které je nutné upravit nebo smazat: / Files which have to be edited or deleted:
%   projekt-20-literatura-bibliography.bib - literatura / bibliography
%   projekt-01-kapitoly-chapters.tex - obsah práce / the thesis content
%   projekt-01-kapitoly-chapters-en.tex - obsah práce v angličtině / the thesis content in English
%   projekt-30-prilohy-appendices.tex - přílohy / appendices
%   projekt-30-prilohy-appendices-en.tex - přílohy v angličtině / appendices in English
%==============================================================================
\documentclass[]{fitthesis} % bez zadání - pro začátek práce, aby nebyl problém s překladem
%\documentclass[english]{fitthesis} % without assignment - for the work start to avoid compilation problem
%\documentclass[zadani]{fitthesis} % odevzdani do wisu a/nebo tisk s barevnými odkazy - odkazy jsou barevné
%\documentclass[english,zadani]{fitthesis} % for submission to the IS FIT and/or print with color links - links are color
%\documentclass[zadani,print]{fitthesis} % pro černobílý tisk - odkazy jsou černé
%\documentclass[english,zadani,print]{fitthesis} % for the black and white print - links are black
%\documentclass[zadani,cprint]{fitthesis} % pro barevný tisk - odkazy jsou černé, znak VUT barevný
%\documentclass[english,zadani,cprint]{fitthesis} % for the print - links are black, logo is color
% * Je-li práce psaná v anglickém jazyce, je zapotřebí u třídy použít 
%   parametr english následovně:
%   If thesis is written in English, it is necessary to use 
%   parameter english as follows:
%      \documentclass[english]{fitthesis}
% * Je-li práce psaná ve slovenském jazyce, je zapotřebí u třídy použít 
%   parametr slovak následovně:
%   If the work is written in the Slovak language, it is necessary 
%   to use parameter slovak as follows:
%      \documentclass[slovak]{fitthesis}
% * Je-li práce psaná v anglickém jazyce se slovenským abstraktem apod., 
%   je zapotřebí u třídy použít parametry english a enslovak následovně:
%   If the work is written in English with the Slovak abstract, etc., 
%   it is necessary to use parameters english and enslovak as follows:
%      \documentclass[english,enslovak]{fitthesis}

% Základní balíčky jsou dole v souboru šablony fitthesis.cls
% Basic packages are at the bottom of template file fitthesis.cls
% zde můžeme vložit vlastní balíčky / you can place own packages here

% Kompilace po částech (rychlejší, ale v náhledu nemusí být vše aktuální)
% Compilation piecewise (faster, but not all parts in preview will be up-to-date)
% \usepackage{subfiles}

% Nastavení cesty k obrázkům
% Setting of a path to the pictures
%\graphicspath{{obrazky-figures/}{./obrazky-figures/}}
%\graphicspath{{obrazky-figures/}{../obrazky-figures/}}

%---rm---------------
\renewcommand{\rmdefault}{lmr}%zavede Latin Modern Roman jako rm / set Latin Modern Roman as rm
%---sf---------------
\renewcommand{\sfdefault}{qhv}%zavede TeX Gyre Heros jako sf
%---tt------------
\renewcommand{\ttdefault}{lmtt}% zavede Latin Modern tt jako tt

% vypne funkci šablony, která automaticky nahrazuje uvozovky,
% aby nebyly prováděny nevhodné náhrady v popisech API apod.
% disables function of the template which replaces quotation marks
% to avoid unnecessary replacements in the API descriptions etc.
\csdoublequotesoff



\usepackage{url}


% =======================================================================
% balíček "hyperref" vytváří klikací odkazy v pdf, pokud tedy použijeme pdflatex
% problém je, že balíček hyperref musí být uveden jako poslední, takže nemůže
% být v šabloně
% "hyperref" package create clickable links in pdf if you are using pdflatex.
% Problem is that this package have to be introduced as the last one so it 
% can not be placed in the template file.
\ifWis
\ifx\pdfoutput\undefined % nejedeme pod pdflatexem / we are not using pdflatex
\else
  \usepackage{color}
  \usepackage[unicode,colorlinks,hyperindex,plainpages=false,pdftex]{hyperref}
  \definecolor{hrcolor-ref}{RGB}{223,52,30}
  \definecolor{hrcolor-cite}{HTML}{2F8F00}
  \definecolor{hrcolor-urls}{HTML}{092EAB}
  \hypersetup{
	linkcolor=hrcolor-ref,
	citecolor=hrcolor-cite,
	filecolor=magenta,
	urlcolor=hrcolor-urls
  }
  \def\pdfBorderAttrs{/Border [0 0 0] }  % bez okrajů kolem odkazů / without margins around links
  \pdfcompresslevel=9
\fi
\else % pro tisk budou odkazy, na které se dá klikat, černé / for the print clickable links will be black
\ifx\pdfoutput\undefined % nejedeme pod pdflatexem / we are not using pdflatex
\else
  \usepackage{color}
  \usepackage[unicode,colorlinks,hyperindex,plainpages=false,pdftex,urlcolor=black,linkcolor=black,citecolor=black]{hyperref}
  \definecolor{links}{rgb}{0,0,0}
  \definecolor{anchors}{rgb}{0,0,0}
  \def\AnchorColor{anchors}
  \def\LinkColor{links}
  \def\pdfBorderAttrs{/Border [0 0 0] } % bez okrajů kolem odkazů / without margins around links
  \pdfcompresslevel=9
\fi
\fi
% Řešení problému, kdy klikací odkazy na obrázky vedou za obrázek
% This solves the problems with links which leads after the picture
\usepackage[all]{hypcap}

% Informace o práci/projektu / Information about the thesis
%---------------------------------------------------------------------------
\projectinfo{
  %Prace / Thesis
  project={SP},            %typ práce BP/SP/DP/DR  / thesis type (SP = term project)
  year={2021},             % rok odevzdání / year of submission
  date=\today,             % datum odevzdání / submission date
  %Nazev prace / thesis title
  title.cs={MAC flooding útok},  % název práce v češtině či slovenštině (dle zadání) / thesis title in czech language (according to assignment)
  title.en={Thesis title}, % název práce v angličtině / thesis title in english
  %title.length={14.5cm}, % nastavení délky bloku s titulkem pro úpravu zalomení řádku (lze definovat zde nebo níže) / setting the length of a block with a thesis title for adjusting a line break (can be defined here or below)
  %sectitle.length={14.5cm}, % nastavení délky bloku s druhým titulkem pro úpravu zalomení řádku (lze definovat zde nebo níže) / setting the length of a block with a second thesis title for adjusting a line break (can be defined here or below)
  %Autor / Author
  author.name={Daniel},   % jméno autora / author name
  author.surname={Olearčin},   % příjmení autora / author surname 
  %author.title.p={Bc.}, % titul před jménem (nepovinné) / title before the name (optional)
  %author.title.a={Ph.D.}, % titul za jménem (nepovinné) / title after the name (optional)
  %Ustav / Department
  department={UPGM}, % doplňte příslušnou zkratku dle ústavu na zadání: UPSY/UIFS/UITS/UPGM / fill in appropriate abbreviation of the department according to assignment: UPSY/UIFS/UITS/UPGM
  % Školitel / supervisor
  supervisor.name={Pavel},   % jméno školitele / supervisor name 
  supervisor.surname={Očenášek},   % příjmení školitele / supervisor surname
  supervisor.title.p={},   %titul před jménem (nepovinné) / title before the name (optional)
  supervisor.title.a={Mgr. Ing. Ph.D.},    %titul za jménem (nepovinné) / title after the name (optional)
  % Klíčová slova / keywords
  keywords.cs={MAC flooding útok}, % klíčová slova v českém či slovenském jazyce / keywords in czech or slovak language
  keywords.en={MAC flooding attack}, % klíčová slova v anglickém jazyce / keywords in english
  %keywords.en={Here, individual keywords separated by commas will be written in English.},
  % Abstrakt / Abstract
  abstract.cs={Cieľom práce bolo zistiť ako funguje MAC flooding útok, čo sa snažíme docieliť a ako na to.}, % abstrakt v českém či slovenském jazyce / abstract in czech or slovak language
  abstract.en={The aim of the work was to find out how the MAC flooding attack works, what we are trying to achieve and how to do it.}, % abstrakt v anglickém jazyce / abstract in english
  %abstract.en={An abstract of the work in English will be written in this paragraph.},
  % Prohlášení (u anglicky psané práce anglicky, u slovensky psané práce slovensky) / Declaration (for thesis in english should be in english)
  declaration={Prohlašuji, že jsem tuto bakalářskou práci vypracoval samostatně a
uvedl jsem všechny literární prameny, publikace a další zdroje, ze kterých jsem čerpal.},
  %declaration={I hereby declare that this Bachelor's thesis was prepared as an original work by the author under the supervision of Mr. X
% The supplementary information was provided by Mr. Y
% I have listed all the literary sources, publications and other sources, which were used during the preparation of this thesis.},
  % Poděkování (nepovinné, nejlépe v jazyce práce) / Acknowledgement (optional, ideally in the language of the thesis)
  %acknowledgment={V této sekci je možno uvést poděkování vedoucímu práce a těm, kteří poskytli odbornou pomoc
%(externí zadavatel, konzultant apod.).},
  %acknowledgment={Here it is possible to express thanks to the supervisor and to the people which provided professional help
%(external submitter, consultant, etc.).},
  % Rozšířený abstrakt (cca 3 normostrany) - lze definovat zde nebo níže / Extended abstract (approximately 3 standard pages) - can be defined here or below
  %extendedabstract={Do tohoto odstavce bude zapsán rozšířený výtah (abstrakt) práce v českém (slovenském) jazyce.},
  %faculty={FIT}, % FIT/FEKT/FSI/FA/FCH/FP/FAST/FAVU/USI/DEF
  faculty.cs={Fakulta informačních technologií}, % Fakulta v češtině - pro využití této položky výše zvolte fakultu DEF / Faculty in Czech - for use of this entry select DEF above
  faculty.en={Faculty of Information Technology}, % Fakulta v angličtině - pro využití této položky výše zvolte fakultu DEF / Faculty in English - for use of this entry select DEF above
  department.cs={Ústav matematiky}, % Ústav v češtině - pro využití této položky výše zvolte ústav DEF nebo jej zakomentujte / Department in Czech - for use of this entry select DEF above or comment it out
  department.en={Institute of Mathematics} % Ústav v angličtině - pro využití této položky výše zvolte ústav DEF nebo jej zakomentujte / Department in English - for use of this entry select DEF above or comment it out
}

% Rozšířený abstrakt (cca 3 normostrany) - lze definovat zde nebo výše / Extended abstract (approximately 3 standard pages) - can be defined here or above
%\extendedabstract{Do tohoto odstavce bude zapsán výtah (abstrakt) práce v českém (slovenském) jazyce.}

% nastavení délky bloku s titulkem pro úpravu zalomení řádku - lze definovat zde nebo výše / setting the length of a block with a thesis title for adjusting a line break - can be defined here or above
%\titlelength{14.5cm}
% nastavení délky bloku s druhým titulkem pro úpravu zalomení řádku - lze definovat zde nebo výše / setting the length of a block with a second thesis title for adjusting a line break - can be defined here or above
%\sectitlelength{14.5cm}

% řeší první/poslední řádek odstavce na předchozí/následující stránce
% solves first/last row of the paragraph on the previous/next page
\clubpenalty=10000
\widowpenalty=10000

% checklist
\newlist{checklist}{itemize}{1}
\setlist[checklist]{label=$\square$}

\begin{document}
  % Vysazeni titulnich stran / Typesetting of the title pages
  % ----------------------------------------------
  \maketitle
  % Obsah
  % ----------------------------------------------
  \setlength{\parskip}{0pt}

  {\hypersetup{hidelinks}\tableofcontents}
  
  % Seznam obrazku a tabulek (pokud prace obsahuje velke mnozstvi obrazku, tak se to hodi)
  % List of figures and list of tables (if the thesis contains a lot of pictures, it is good)
  \ifczech
    \renewcommand\listfigurename{Seznam obrázků}
  \fi
  \ifslovak
    \renewcommand\listfigurename{Zoznam obrázkov}
  \fi
  % {\hypersetup{hidelinks}\listoffigures}
  
  \ifczech
    \renewcommand\listtablename{Seznam tabulek}
  \fi
  \ifslovak
    \renewcommand\listtablename{Zoznam tabuliek}
  \fi
  % {\hypersetup{hidelinks}\listoftables}

  \ifODSAZ
    \setlength{\parskip}{0.5\bigskipamount}
  \else
    \setlength{\parskip}{0pt}
  \fi

  % vynechani stranky v oboustrannem rezimu
  % Skip the page in the two-sided mode
  \iftwoside
    \cleardoublepage
  \fi

  % Text prace / Thesis text
  % ----------------------------------------------
  \ifenglish
    \input{projekt-01-kapitoly-chapters-en}
  \else
    % Tento soubor nahraďte vlastním souborem s obsahem práce.
%=========================================================================
% Autoři: Michal Bidlo, Bohuslav Křena, Jaroslav Dytrych, Petr Veigend a Adam Herout 2019
\chapter{Úvod}
\section{Čo je to GDPR}
Všeobecné nariadenie o ochrane údajov (anglicky General Data Protection Regulation, skrátene GDPR), plným názvom Nariadenie Európskeho parlamentu a Rady (EÚ) 2016/679 z 27. apríla 2016 o ochrane fyzických osôb pri spracúvaní osobných údajov a o voľnom pohybe takýchto údajov, ktorým sa zrušuje smernica 95/46/ES (všeobecné nariadenie o ochrane údajov), je nariadenie Európskej únie, ktorého cieľom je výrazné zvýšenie ochrany osobných údajov občanov. V Úradnom vestníku Európskej únie bolo vyhlásené 27. apríla 2016. Nariadenia, v schválenom znení, sú priamo účinné pre každý členský štát EÚ. V slovenskom právnom poriadku je nariadenie premietnuté aj do zákona č. 18/2018 Z. z. o ochrane osobných údajov a o zmene a doplnení niektorých zákonov z 29. novembra 2017. Zákon 18/2018 Z. z. je účinný na Slovensku rovnako ako GDPR od 25. mája 2018. Viac viz \cite{wikiGDPR}.
\section{Prečo vzniklo GDPR}
Nariadenie chráni základné práva a slobody fyzických osôb so zameraním na právo ochrany osobných údajov. Stanovuje pravidlá ochrany fyzických osôb a pravidlá pre voľný pohyb (ktorý sa nesmie obmedziť ani zakázať) osobných údajov v Európskej únii. Táto konzistentná úroveň ochrany fyzických osôb v celej Únii má zabrániť rozdielom, ktoré sú prekážkou voľného pohybu osobných údajov v rámci vnútorného trhu (preambula, ods.13). Viac viz \cite{wikiGDPR}.
\newpage
\section{Na čo všetko sa vzťahuje GDPR}
Nariadenie spresňuje a rozširuje okruh a definíciu osobných údajov takto: akékoľvek informácie týkajúce sa identifikovanej alebo identifikovateľnej fyzickej osoby (ďalej len „dotknutá osoba“); identifikovateľná fyzická osoba je osoba, ktorú možno identifikovať priamo alebo nepriamo, najmä odkazom na identifikátor, ako je meno, identifikačné číslo, lokalizačné údaje, online identifikátor, alebo odkazom na jeden či viaceré prvky, ktoré sú špecifické pre fyzickú, fyziologickú, genetickú, mentálnu, ekonomickú, kultúrnu alebo sociálnu identitu tejto fyzickej osoby. Osobným údajom je teda aj emailová adresa, dokonca podľa nariadenia GDPR osobné údaje (online identifikátory) sú aj cookies,IP adresy, či iné virtuálne identifikátory.

Osobné údaje sú informácie o osobe, preto osobným údajom nie sú napr. údaje o právnickej osobe (o jej zamestnancoch už ale áno), údaje o osobách zomretých, nie sú to údaje, ktoré konkrétnu osobu nestotožňujú (napr. Obyčajné bežné meno a priezvisko) a medzi osobné údaje nepatria údaje anonymizované, teda také, ktoré pôvodne možnosť identifikácie osoby obsahovali, ale taký identifikátor z nich bol odstránený. Viac viz \cite{wikiGDPR}.

\chapter{Všeobecné technické požiadavky na zabezpečenie podľa GDPR}
\section{Všeobecné nariadenie o ochrane údajov}
Všeobecné nariadenie o ochrane údajov sa uplatňuje, keď:
\begin{itemize}
\item vaša spoločnosť spracúva osobné údaje a má sídlo v EÚ bez ohľadu na to, kde sa skutočné spracúvanie údajov uskutočňuje,
\item vaša spoločnosť je usadená mimo EÚ, ale spracúva osobné údaje v súvislosti s ponukou tovaru alebo služieb jednotlivcom v EÚ alebo monitoruje správanie jednotlivcov v rámci EÚ.
\end{itemize}
Podniky, ktoré nemajú sídlo v EÚ a spracúvajú údaje občanov EÚ, musia vymenovať zástupcu v EÚ.\\

Všeobecné nariadenie o ochrane údajov sa neuplatňuje, keď:
\begin{itemize}
\item dotknutá osoba je mŕtva,
\item dotknutá osoba je právnickou osobou, jednotlivcov v rámci EÚ.
\item spracúvanie sa vykonáva osobou konajúcou na účely, ktoré nesúvisia s jej obchodnou činnosťou, podnikaním alebo povolaním.
\end{itemize}
Viac viz \cite{strankaP}.
\section{Spracovávanie osobných údajov v rámci firmy}
Monitorovaním spracúvania osobných údajov a poskytovaním informácií a poradenstva zamestnancom, ktorí spracúvajú osobné údaje, v súvislosti s ich povinnosťami je poverená zodpovedná osoba, ktorá môže byť určená spoločnosťou. Zodpovedná osoba takisto spolupracuje s orgánom pre ochranu osobných údajov, pričom slúži ako kontaktné miesto pre orgán pre ochranu osobných údajov a jednotlivcov.
\newpage
Vaša spoločnosť je povinná vymenovať zodpovednú osobu, ak:
\begin{itemize}
\item pravidelne alebo systematicky monitoruje jednotlivcov alebo spracúva osobitné kategórie údajov,
\item toto spracúvanie je hlavnou podnikateľskou činnosťou,
\item spracúva údaje vo veľkom rozsahu.
\end{itemize}
Napríklad, ak spracúvate osobné údaje s cieľom zacieliť reklamu prostredníctvom vyhľadávačov na základe správania ľudí na internete, vyžaduje sa, aby ste určili zodpovednú osobu. Ak však iba raz ročne posielate klientom propagačné materiály, nepotrebujete mať zodpovednú osobu. Podobne, ak ste lekár a získavate údaje o zdraví pacientov, pravdepodobne nepotrebujete mať zodpovednú osobu. Ak však spracúvate osobné údaje týkajúce sa genetiky alebo zdravia pre nemocnicu, potom sa vyžaduje mať zodpovednú osobu.

Zodpovedná osoba môže byť pracovníkom vašej organizácie alebo ňou môže byť externá osoba na základe zmluvy o poskytovaní služieb. Zodpovedná osoba môže byť jednotlivcom alebo súčasťou organizácie. Viac viz \cite{strankaP}.
\section{Kedy sa môžu spracovávať osobne údaje}
Na základe pravidiel EÚ o ochrane údajov by ste mali údaje spracúvať spravodlivo a zákonným spôsobom na určené a legitímne účely a spracúvať len údaje, ktoré sú potrebné na splnenie tohto účelu. Na spracúvanie osobných údajov sa vyžaduje, aby ste spĺňali jednu z týchto podmienok:
\begin{itemize}
\item máte súhlas dotknutej osoby,
\item osobné údaje potrebujete na plnenie zmluvných záväzkov voči dotknutej osobe,
\item osobné údaje potrebujete na splnenie právnej povinnosti,
\item osobné údaje potrebujete na ochranu životne dôležitých záujmov dotknutej osoby,
\item osobné údaje spracúvate na účely plnenia úlohy vo verejnom záujme,
\item konáte v oprávnených záujmoch svojej spoločnosti, pokiaľ sa tým závažným spôsobom neovplyvňujú základné práva a slobody osoby, ktorej údaje sa spracúvajú. Ak práva dotknutej osoby prevažujú nad záujmami vašej spoločnosti, dané osobné údaje nemôžete spracúvať.
\end{itemize}
Viac viz \cite{strankaP}.
\newpage
\section{Poskytovanie transparentných informácií}
Jednotlivcom musíte poskytnúť jasné informácie o tom, kto osobné údaje o nich spracúva a prečo. Informácie musia zahŕňať minimálne toto:
\begin{itemize}
\item kto ste,
\item prečo spracúvate osobné údaje,
\item aký je právny základ,
\item (prípadne) kto získa tieto údaje.
\end{itemize}
V niektorých prípadoch musia poskytované informácie obsahovať aj:
\begin{itemize}
\item kontaktné údaje prípadnej zodpovednej osoby,
\item aký oprávnený záujem spoločnosť sleduje, ak je to právnym dôvodom na spracovanie údajov,
\item opatrenia uplatňované na prenos údajov do krajiny mimo EÚ,
\item práva na ochranu údajov jednotlivca (t. j. právo na prístup, opravu, vymazanie, obmedzenie, námietku, prenosnosť atď.)
\item a mnoho ďalších.
\end{itemize}
Viac viz \cite{strankaP}.
\newpage
\chapter{GDPR v rámci mobilných tenchológií.}
\section{GDPR proti ponukám cez telefón}
S nariadením o ochrane osobných údajov sú firmy opatrnejšie pri ponúkaní produktov cez telefón alebo SMS.  GDPR sa snaží pri tomto procese zaistiť väčšiu ochranu nad osobnými údajmi. Kto vám môže ponúkať takéto služby a ako sa vyhnúť nepríjemným telefonátom? Ak už nechcete, aby vám niečo ponúkali, jednoducho ich na to upozornite. Viac viz \cite{mobily}.
\section{Kto vám môže ponúkať svoje služby a produkty po telefóne}
GDPR dopadá na telefonické ponuky a ponuky cez SMS podobne ako na emailové reklamné oznámenia. Ako vaše telefónne číslo, tak aj emailová adresa sú osobným údajom. Spoločnosti môžu svoje produkty či služby po telefóne, cez SMS alebo emailom ponúkať len na základe vášho súhlasu. Ak si tak kúpite nový mobilný telefón, je predajca oprávnený vám zaslať napríklad ponuku príslušenstva k vybranému prístroju. Ďalej sa môže stať, že vás osloví marketingová agentúra, s ktorou má správca údajov uzatvorenú zmluvu, aby vám služby či produkty daného správcu ponúkala. Pokiaľ vás ale osloví niekto s ponukou bez toho, aby ste s danou spoločnosťou mali nejaký vzťah, môže ísť o porušenie predpisov o ochrane osobných údajov. Viac viz \cite{mobily}.
\section{Ako je to so spoločnosťami, ktoré si dáta o zákazníkoch kupujú}
Spoločnosti, ktoré prevádzkujú telemarketing, zvyčajne pracujú s telefónnymi číslami, ktoré nezískavajú priamo od ľudí, ale z databáz s údajmi. S databázami údajov tohto druhu sa často obchoduje a dáta sa do nich získavajú tak, že zákazníci e-shopov, užívatelia rôznych webov a ďalších služieb zadajú svoje telefónne čísla a odsúhlasia podmienky spracovania osobných údajov, v ktorých je ukrytý súhlas s takýmto využitím ich mobilného čísla. Viac viz \cite{mobily}.
\section{Je dôležitý jasný súhlas}
GDPR zavádza pre súhlas so spracovaním osobných údajov niektoré nové podmienky – v tomto prípade je najmä dôležitý dôraz na odlišnosť súhlasu od ostatných skutočností.

Subjekt musí jasne vedieť, že udeľuje súhlas so spracovaním údajov a súhlasom nesmie byť podmienené napríklad plnenie zmluvy. Vzhľadom k tomu, že súhlasy ukryté v obchodných podmienkach tieto kritériá nespĺňajú, nemožno údaje získané takouto formou naďalej využívať. Aby to bolo možné, musel by byť súhlas udelený znovu podľa podmienok, ktoré požaduje GDPR. Viac viz \cite{mobily}.
\section{Ako sa nechať vymazať?}
Platí, že zákazník musí mať možnosť jednoducho (napr. prostredníctvom odkazu) bezplatne sa odhlásiť, resp. odmietnuť ďalšie zasielanie marketingových oznámení. Jedná sa typicky o akékoľvek obchodné oznámenia a ponuky zasielané na email alebo telefónne číslo zákazníka. GDPR nestanovuje jednotný postup pre uplatnenie práva na to nebyť ďalej oslovovaný s reklamnými ponukami. Jeho uplatnenie by ale malo byť rovnako jednoduché, ako udelenie súhlasu. Uplatnenie tohto práva sa tak líši spoločnosť od spoločnosti.

Všeobecne je možné podať takúto žiadosť napr. listom, emailom, cez web. V prípade telefonického oslovovania bude najskôr rovnako možné volajúcemu oznámiť žiadosť o výmaz. Ľudia by si však mali byť vedomí, že uplatnením práva na vymazanie nemusí dôjsť k vymazaniu všetkých ich osobných údajov. Keď napríklad odoberú operátorovi súhlas pre marketingové akcie, ten bude aj naďalej spravovať ich údaje k zmluve napríklad o tarife. Pravdepodobne dôjde k vymazaniu údajov používaných pre marketingové oslovovanie, ale nemusí dôjsť k vymazaniu údajov, ktoré spoločnosť potrebuje na plnenie zákonných povinností či plnenie uzatvorenej zmluvy. Viac viz \cite{mobily}.
\section{Je s GDPR vymazanie z marketingových databáz jednoduchšie?}
Takúto možnosť poznala aj doterajšiu právna úprava, GDPR však celkovo zvyšuje dôraz na dodržiavanie pravidiel ochrany osobných údajov, takže možno očakávať, že k právam dotknutých osôb budú mať spoločnosti väčší rešpekt. Všeobecne je GDPR ústretové k elektronickej forme komunikácie, takže možno komunikovať emailom, ale je na mieste si všetku komunikáciu uchovávať. Viac viz \cite{mobily}.

 
%===============================================================================

  \fi
  
  % Kompilace po částech (viz výše, nutno odkomentovat)
  % Compilation piecewise (see above, it is necessary to uncomment it)
  %\subfile{projekt-01-uvod-introduction}
  % ...
  %\subfile{chapters/projekt-05-conclusion}


  % Pouzita literatura / Bibliography
  % ----------------------------------------------
  \newpage
  \bibliographystyle{czechiso}
  \bibliography{projekt-20-literatura-bibliography}

  % vynechani stranky v oboustrannem rezimu
  % Skip the page in the two-sided mode
  \iftwoside
    \cleardoublepage
  \fi

  % Prilohy / Appendices
  % ---------------------------------------------
  \appendix
\ifczech
  \renewcommand{\appendixpagename}{Přílohy}
  \renewcommand{\appendixtocname}{Přílohy}
  \renewcommand{\appendixname}{Příloha}
\fi
\ifslovak
  \renewcommand{\appendixpagename}{Prílohy}
  \renewcommand{\appendixtocname}{Prílohy}
  \renewcommand{\appendixname}{Príloha}
\fi
%  \appendixpage

% vynechani stranky v oboustrannem rezimu
% Skip the page in the two-sided mode
%\iftwoside
%  \cleardoublepage
%\fi
  
\ifslovak
%  \section*{Zoznam príloh}
%  \addcontentsline{toc}{section}{Zoznam príloh}
\else
  \ifczech
%    \section*{Seznam příloh}
%    \addcontentsline{toc}{section}{Seznam příloh}
  \else
%    \section*{List of Appendices}
%    \addcontentsline{toc}{section}{List of Appendices}
  \fi
\fi
  \startcontents[chapters]
  \setlength{\parskip}{0pt} 
  % seznam příloh / list of appendices
  % \printcontents[chapters]{l}{0}{\setcounter{tocdepth}{2}}
  
  \ifODSAZ
    \setlength{\parskip}{0.5\bigskipamount}
  \else
    \setlength{\parskip}{0pt}
  \fi
  
  % vynechani stranky v oboustrannem rezimu
  \iftwoside
    \cleardoublepage
  \fi
  
  % Přílohy / Appendices
  %\ifenglish
    %\input{projekt-30-prilohy-appendices-en}
  %\else
    %\input{projekt-30-prilohy-appendices}
  %\fi
  
  % Kompilace po částech (viz výše, nutno odkomentovat)
  % Compilation piecewise (see above, it is necessary to uncomment it)
  %\subfile{projekt-30-prilohy-appendices}
  
\end{document}
