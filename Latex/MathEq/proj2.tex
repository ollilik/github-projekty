\documentclass[a4paper, 11pt, twocolumn] {article}

\usepackage{times}
\usepackage[IL2]{fontenc}
\usepackage[czech]{babel}
\usepackage[utf8]{inputenc}
\usepackage[left=1.5cm,text={18cm, 25cm},top=2.5cm]{geometry}
\usepackage{amsthm}
\usepackage{amssymb}
\usepackage{mathtools}
\usepackage{xfrac}


\theoremstyle{definition}
\newtheorem{defi}{Definice}

\theoremstyle{plain}
\newtheorem{veta}{Věta}


\begin{document}
	\begin{titlepage}
		\begin{center}
			\huge{FAKULTA INFORMAČNÍCH TECHNOLOGIÍ\\
			VYSOKÉ UČENÍ TECHNICKÉ V BRNĚ}\\
			\vspace{\stretch{0.382}}
			\LARGE{Typografie a publikování – 2. projekt\\
			Sazba dokumentů a matematických výrazů}\\
			\vspace{\stretch{0.618}}
		\end{center}
		\Large{2021 \hfill Daniel Olearčin (xolear00)}
	\end{titlepage}
\setcounter{page}{1}

\section*{Úvod}
V této úloze si vyzkoušíme sazbu titulní strany, matematických vzorců, prostředí a~dalších textových struktur obvyklých pro technicky zaměřené texty (například rovnice (\ref{rovnice1}) nebo Definice \ref{definice1} na straně \pageref{definice1}). Rovněž si vyzkoušíme používání odkazů \verb|\ref| a \verb|\pageref|.

Na titulní straně je využito sázení nadpisu podle optického středu s využitím zlatého řezu. Tento postup byl
probírán na přednášce. Dále je použito odřádkování se
zadanou relativní velikostí 0.4 em a 0.3 em.

V případě, že budete potřebovat vyjádřit matematickou
konstrukci nebo symbol a nebude se Vám dařit jej nalézt
v samotném \LaTeX u, doporučuji prostudovat možnosti balíku maker \AmS-\LaTeX .

\section{Matematický text}
Nejprve se podíváme na sázení matematických symbolů
a výrazů v plynulém textu včetně sazby definic a vět s využitím balíku \texttt{amsthm}. Rovněž použijeme poznámku pod
čarou s použitím příkazu \verb|\footnote| . Někdy je vhodné
použít konstrukci \verb|\mbox{}|, která říká, že text nemá být
zalomen.\\

\begin{defi}
\label{definice1}
Rozšířený zásobníkový automat (RZA) je definován jako sedmice tvaru A = (\(Q, \Sigma, \Gamma, \delta, q_0, Z_0, F\)),
kde:
\begin{itemize}
	\setlength\itemsep{0.1em}
	\item Q je konečná množina vnitřních (řídicích) stavů,
	\item $\Sigma$ je konečná vstupní abeceda,
	\item $\Gamma$ je konečná zásobníková abeceda,
	\item $\delta$ je přechodová funkce $Q\times(\Sigma \cup \{\epsilon\})\times\Gamma^*\to2^{Q\times\Gamma^*}$,
	\item $q_0 \in Q$ je počáteční stav, $Z_0 \in \Gamma$ je startovací symbol
	zásobníku a $F \subseteq Q$ je množina koncových stavů.
\end{itemize}

Nechť P = $(Q, \Sigma, \Gamma, \delta, q_0, Z_0, F)$ je rozšířený zásobníkový automat. Konfigurací nazveme trojici (q, $\omega, \alpha$) $\in$
$Q~\times~\Sigma^*~\times~\Gamma^*$, kde q je aktuální stav vnitřního řízení,
$\omega$ je dosud nezpracovaná část vstupního řetězce a $\alpha$ =
$Z_{i1}Z_{i2}$\dots $Z_{ik}$ je obsah zásobníku\footnote{$Z_{i_1}$
je vrchol zásobníku}.
\end{defi}

\subsection{Podsekce obsahující větu a odkaz}
\begin{defi}
\label{definice2} 
Řetězec $\omega$ nad abecedou $\Sigma$ je přijat RZA
A jestliže ($q_0, \omega, Z_0$) ${\displaystyle \vdash_A^*}$ ($q_F,\epsilon,\gamma$) pro nějaké $\gamma \in \Gamma^*$ a $q_F \in F.$ Množinu L(A) = \{$\omega \mid \omega$ je přijat RZA A\} $\subseteq \Sigma^*$ nazýváme jazyk přijímaný RZA A.\\
$$
f^{\prime}(x)=\lim _{\Delta x \rightarrow 0} \frac{f(x)-f(x+\Delta x)}{\Delta x}
$$

Nyní si vyzkoušíme sazbu vět a důkazů opět s použitím
balíku \texttt{amsthm}.\\
\vspace{-1.5em}
\begin{veta}Třída jazyků, které jsou přijímány ZA, odpovídá
			bezkontextovým jazykům.\\
\end{veta}
\vspace{-2.5em}
\begin{flushleft}Důkaz. V důkaze vyjdeme z Definice \ref{definice1} a 								 \ref{definice2}.\hfill $\square$
\end{flushleft}
\end{defi}


\section{Rovnice a odkazy}
Složitější matematické formulace sázíme mimo plynulý
text. Lze umístit několik výrazů na jeden řádek, ale pak je
třeba tyto vhodně oddělit, například příkazem \verb|\quad|.\\\\
$\sqrt[i]{x^3_i} \quad$ kde $x_i$ je i-té sudé číslo splňující 
$x_i^{x_i^{i^2}+ 2} \leq y_i^{x_i^4}$\\
V rovnici (\ref{rovnice1}) jsou využity tři typy závorek s různou
explicitně definovanou velikostí.\\

\begin{eqnarray}
	\label{rovnice1}
	x & = & \bigg[\Big\{\big[a + b\big]*c\Big]^d \oplus 								2\bigg]^{\sfrac{3}{2}}
\end{eqnarray}
$\quad\quad\quad\quad\quad y \quad = \quad \lim\limits_{x \to \infty} \frac{\displaystyle \frac{1}{\log_{10} x}}{\displaystyle sin^2 x + cos^2 x}$\\

V této větě vidíme, jak vypadá implicitní vysázení limity $\lim_{n\rightarrow\infty} f(n)$ v normálním odstavci textu. Podobně je to i s dalšími symboly jako $\prod\nolimits_{i=1}^{n}2^i$ či $\bigcap_{A\in\mathcal{B}} A$. V~případě vzorců $\lim\limits_{n \to \infty}f(n) \text{ a } \prod\limits_{i=1}^{n}2^i$ jsme si vynutili méně úspornou sazbu příkazem \verb|\limits|.\\
\begin{eqnarray}
	\label{rovnice2}
	\displaystyle\int_{b}^{a} g(x)\: \mathrm{d}x\quad = \quad-\int_{a}^{b} f(x)\: 	\mathrm{d}x 
\end{eqnarray}
\section{Matice}
Pro sázení matic se velmi často používá prostředí \texttt{array}
a závorky (\verb|\left|, \verb|\right|).\\
\begin{center}
$\left(\begin{array}{ccc} a-b \quad \widehat{\xi + \omega} \quad \pi \\
       \quad\vec{\mathbf{a}}\quad\quad \overleftrightarrow{AC}\quad\quad\!\! 			   \hat{\beta} \end{array}
 \right) = 1 \Longleftrightarrow \mathbb{Q} = \mathcal{R}$

$A = 
\begin{Vmatrix}
a_{11} & a_{12} & \cdots & a_{1n} \\
a_{21} & a_{22} & \cdots & a_{2n} \\
\vdots  & \vdots  & \ddots & \vdots  \\
a_{m1} & a_{m2} & \cdots & a_{mn} 
\end{Vmatrix} = 
\begin{vmatrix}
t & u\\
v & w
\end{vmatrix} = 
tw - uv$\\
\end{center}

Prostředí \texttt{array} lze úspěšně využít i jinde.\\
\begin{center}
$\displaystyle\binom{n}{k} = \left\{ \begin{array}{c l}0 & \mbox{pro $k < 0$ nebo $k > n$}\\
        		\displaystyle\frac{n!}{k!(n-k)!} & \mbox{pro $0 \leq k \leq n$}.
    			\end{array} 
    			\right.$
\end{center}
\end{document}