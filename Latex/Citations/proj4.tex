
\documentclass[a4paper, 11pt]{article}
\usepackage[utf8]{inputenc}
\usepackage[top=3cm, text={17cm,24cm}, left=2cm]{geometry}
\usepackage{times}
\usepackage[czech]{babel}
\usepackage[breaklinks]{hyperref}
\usepackage{breakurl}

\begin{document}
	\begin{titlepage}
		\begin{center}
			\huge{VYSOKÉ UČENÍ TECHNICKÉ V BRNĚ}\\
			\LARGE{FAKULTA INFORMAČNÍCH TECHNOLOGIÍ}\\
			\vspace{\stretch{0.382}}
			\Large{Typografie a publikování - 4. projekt}\\
			\huge{Latex - základné informácie}\\
			\vspace{\stretch{0.618}}
		\end{center}
		\Large{\today \hfill Daniel Olearčin}
	\end{titlepage}
\setcounter{page}{1}
\section{Latex}
Jak uvádí článek \cite{Davidek},
za LaTeX (čti „lejtech“ nebo „latech“, zkratka od Lamport TeX ‒ Leslie Lamport je autor) lze v širší rovině považovat typografický systém, který je primárně určen k sazbě vědeckých a matematických dokumentů vysoké typografické kvality. Tento systém je každopádně vhodný i pro tvorbu rozličných dokumentů jako jsou například jednoduché vizitky nebo několikasetstránkové knihy.
\section{Historie latexu}
V historii byl nejdříve vytvořen systém TeX (tvůrce je pan Donald E. Knuth) - tento systém dovoluje vytvářet v maximální míře různé detaily - je to soubor "malých krabiček" - malých funkcí. Dá se s ním pracovat, ale ne dostatečně pohodlně. Pro vyřešení tohoto problému byl vytvořen LaTeX, který nabízí uživateli podstatně "větší krabičky". viz \cite{Frcatel}
\section{Informace o latexu}
Začátečníkovi v \TeX u lze vřele doporučit např. publikaci LATEX pro začátečníky viz \cite{Rybicka} nebo Jemný úvod do \TeX u \cite{Doob}. Pro již zkušenější uživatele bude zajisté velmi užitečný manuál \TeX book \cite{Knuth}
\section{Texmaker}
Tato práce byla napsána v TeXmakeru. TeXmaker je pro uživatele jeden z nejpřívìtivìjších editorú vúbec, alespoň z mého pohledu, neboť mì se v nìm pracovalo nejlépe. Viac viz \cite{Bartlik} alebo \cite{texmaker}
\section{Manuály}
Existuje mnoho manuálů které nám pomáhají zacházet s latexem. Většinou jsou v angličtině viz \cite{manen1} alebo \cite{manen2} ale najdeme i v češtině viz \cite{mancz}

\newpage
\bibliographystyle{czechiso}
\bibliography{proj4}
\end{document}